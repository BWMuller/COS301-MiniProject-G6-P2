\documentclass{article}

\title{Web application}
\author{}

\begin{document}
\maketitle

\section{Web Application}
The web application of the Squirrel Marking System provides a browser-based user interface in the application.

\subsection{Lower Levels of Granularity}
The lower levels of granularity consists of 3 main users:

Student : This user is first required to log in before he/she can use the system. The only functions  permitted to the student to perform on this system is to view his registered modules' 	marks and render his marks onto the web screen, CSV or PDF file.

Marker : This user is also required to log in before he/she can use the system. A marker can be a 	lecturer or student. A marker can only mark leaf or aggregate assessments of certain students. He can submit,add,modify and delete students' marks.

Lecturer : This user is also required to log in before he/she can use the system. The lecturer is by default also a marker.The lecturer can define and control assessments,that is he/she can   		create,modify and delete assessments as well as create, modify, lock and unlock assessment sessions. He/she can also generate reports for a specified assessment, specified student or audit log , having it rendered on the web user interface, CSV or PDF document and publish students' marks.


\end{document}


